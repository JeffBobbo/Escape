\documentclass[english]{article}
\usepackage{caption}
\usepackage{geometry}
\usepackage{graphicx}
\geometry{verbose,tmargin=2cm,bmargin=2cm,lmargin=2cm,rmargin=2cm}

\def \imgres {1000} % dpi
\def \maximgwidth {0.9\textwidth}     % percentage of text-area width to allow images to occupy
\def \maximgheight {0.4\textheight}   % percentage of text-area height to allow images to occupy
\newcommand{\sociablyinclude}[1]{\centering\includegraphics[width=\maximgwidth, height=\maximgheight, keepaspectratio, resolution=\imgres]{#1}}


\title{Escape\\ User Manual}
\author{} % no author
\date{} % no date

\begin{document}
\pagenumbering{gobble}
\maketitle
\tableofcontents

\sociablyinclude{screenshot}
\section{Objective}
Held captive in a poorly constructed cell, your goal is to escape!
Your performance will be monitored by the installation of cameras.
Your progress will be stalled by sentries and guards

\section{Controls}
\begin{table}[!h]
\caption*{}
\label{tab:control}
\begin{tabular}{|c|c|}
\hline
Key & Action \\
\hline
`A' \& `D' & Move (left and right) \\
\hline
`E' & Use (perform action) \\
\hline
`SPACE' & Jump \\
\hline
`Q' & Walk modifier, move at half speed \\
\hline
\end{tabular}
\end{table}
Controls can be modified and saved in a file called `controls.txt'.

\section{Debugging}
`Escape' supports a small handful of command line arguments to assist with debugging.

\begin{table}[!h]
\caption*{}
\label{tab:control}
\begin{tabular}{|c|c|}
\hline
Flag & Effect \\
\hline
``-l=level'' & Loads `level' instead of the main menu. \\
\hline
``-v'' & Draws red bounding volumnes used for collision detection. \\
\hline
``-b'' & Draws green boxes showing the size of objects. \\
\hline
``-p'' & Draws the paths of moving platforms. \\
\hline
\end{tabular}
\end{table}

\end{document}
